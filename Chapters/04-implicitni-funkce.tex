\documentclass[../main.tex]{subfiles}

\begin{document}

%%%%%%%%%%%%%%%%%%%%%%%%%%%%%%%%%%%%%%%%%%%%%%%%%%%%%%%%%%%%%%%%%%%%%%%%%%%%%%%%%%%%%%%%%%%%%%%%%%%%%%%%%
\section{Implicitní funkce}
\hspace{1.2mm}
\noindent

%%%%%%%%%%%%%%%%%%%%%%%%%%%%%%%%%%%%%%%%%%%%%%%%%%%%%%%%%%%%%%%%%%%%%%%%%%%%%%%%%%%%%%%%%%%%%%%%%%%%%%%%%
\subsection{Ilustrační příklady}

\subsubsection{Obecný příklad}
\noindent
\hspace{1.2mm}
Mějme spojité reálné funkce $F_i(x_1, ... , x_m, y_1, ... , y_n)$ pro každé $i \in \{1, ..., n\}$
v $n + m$
proměnných. Určuje systém rovnic
\[ F_1(x_1, ... , x_m, y_1, ... , y_n) = 0 \]
\[ \vdots \hspace{15mm} \vdots \hspace{15mm} \vdots \]
\[ F_n(x_1, ... , x_m, y_1, ... , y_n) = 0 \]
v nějakém smyslu funkce
\[ f_i \equiv y_i(x_1, ... , x_m) \]
pro $i \in \{ 1, ... , n \}$? Pokud ano, jak a kde je určuje a jaké mají funkce vlastnosti?

\vspace{5mm}
\noindent
\hspace{1.2mm}
Konkrétněji viz následující příklad.

\subsubsection{Příklad pro $F(x,y) = x^2 + y^2 - 1$}
\noindent
\hspace{1.2mm}
Mějme $F(x,y) = x^2 + y^2 - 1$, neboli rovnici \[ x^2 + y^2 = 1 \]

\noindent
\hspace{1.2mm}
Několik pozorování:
\begin{itemize}
    \item Pro některá $x_0$ jako například $x_0 < -1$ řešení neexistuje, o funkci $y(x)$ nemluvě.
    \item Přestože řešení v nějakém okolí $x_0$ existuje, nemůžeme v nějakých situacích hovořit o funkci.
    Potřebujeme kolem řešení $(x_0, y_0)$ vymezit okolí jak $x_0$, tak $y_0$.
    \item Máme také případy, jako ten, kdy $x_0 = 1$, kde je v okolí mnoho řešení, ale žádný(ani
    jednostranný) interval, kde by $y$ bylo jednoznačné.
\end{itemize}

\noindent
\hspace{1.2mm}
V případě $F(x,y)$ už zádná další situace nenastane.

%%%%%%%%%%%%%%%%%%%%%%%%%%%%%%%%%%%%%%%%%%%%%%%%%%%%%%%%%%%%%%%%%%%%%%%%%%%%%%%%%%%%%%%%%%%%%%%%%%%%%%%%%
\subsection{Věta o implicitní funkci}
\hspace{1.2mm}
\noindent
Buď $F(x,y)$ reálná funkce definovaná v nějakém okolí bodu $(x_0, y_0)$. Nechť má $F$ spojité parciální
derivace do řádu $k \geq 1$ a nechť platí:
\begin{align*}
    F(x_0, y_0) &= 0\\
    \left| \frac{\partial F(x_0,y_0)}{\partial y} \right| &\neq 0
\end{align*}
Potom $ \exists \delta > 0$ a $\Delta > 0$ takové, že
$\forall x \in (x_0 - \delta , x_0 + \delta) \exists! y \in (y_0 - \Delta , y_0 + \Delta): F(x,y) = 0$.

\noindent
Dále, označíme-li toto jediné $y$ jako $y = f(x)$, potom získaná
$f: (x_0 - \delta , x_0 + \delta ) \to \mathbb{R}$ má spojité derivace do řádu $k$.

%%%%%%%%%%%%%%%%%%%%%%%%%%%%%%%%%%%%%%%%%%%%%%%%%%%%%%%%%%%%%%%%%%%%%%%%%%%%%%%%%%%%%%%%%%%%%%%%%%%%%%%%%
% Tohle tam psat nebudu, neni to uplne necessary a neukaze ti to nic uplne novyho
%\subsection{Definice (implicitní?) funkce a vlastnosti}
%\hspace{1.2mm}
%(5. str 7)
%\noindent

%%%%%%%%%%%%%%%%%%%%%%%%%%%%%%%%%%%%%%%%%%%%%%%%%%%%%%%%%%%%%%%%%%%%%%%%%%%%%%%%%%%%%%%%%%%%%%%%%%%%%%%%%
\subsection{Věta o implicitních funkcích}
\hspace{1.2mm}
\noindent
Buďte $F_i(\mathbf{x}, y_1, ... , y_m)$ pro $i \in {1, ... , m}$ funkce $n+m$ proměnných se spojitými
parciálními derivacemi do řádu $k \geq 1$. Buď \[ \mathbf{F}(\mathbf{x}^0, \mathbf{y}^0) = \mathbf{o} \]
a buď totální diferenciál v bodě $(\mathbf{x}^0, \mathbf{y}^0)$ \[ \frac{D(\mathbf{F})}{D(\mathbf{y})}(\mathbf{x}^0, \mathbf{y}^0) \neq 0 \]
Potom existují $\delta > 0$ a $\Delta > 0$ takové, že pro každé
\[ \mathbf{x} \in (x_{1}^{0} - \delta, x_{1}^{0} + \delta) \times \cdot \cdot \cdot \times 
(x_{n}^{0} - \delta, x_{n}^{0} + \delta)\]
existuje právě jedno
\[ \mathbf{y} \in (y_{1}^{0} - \Delta , y_{1}^{0} + \Delta) \times \cdot \cdot \cdot \times
(y_{m}^{0} - \Delta , y_{m}^{0} + \Delta) \]
takové, že
\[ \mathbf{F}(\mathbf{x}, \mathbf{y}) = 0 \]

%%%%%%%%%%%%%%%%%%%%%%%%%%%%%%%%%%%%%%%%%%%%%%%%%%%%%%%%%%%%%%%%%%%%%%%%%%%%%%%%%%%%%%%%%%%%%%%%%%%%%%%%%
\subsection{Definice Jacobiho determinantu}
\hspace{1.2mm}
\noindent
Pro konečnou posloupnost funkcí
\[ \mathbf{F}(\mathbf{x}, \mathbf{y}) =
(F_1(\mathbf{x}, y_1, ..., y_m), ... , F_m(\mathbf{x}, y_1, ..., y_m)) \]
a pro $\mathbf{y} = (y_1, ... , y_m)$ se definuje \textbf{Jacobiho determinant}(Jacobián) jako
\[ \frac{D(\mathbf{F})}{D(\mathbf{y})} =
\det \left( \frac{\partial F_i}{\partial y_j} \right)_{i,j \in \{ 1, ... , m\}} \]

\end{document}