\documentclass[../main.tex]{subfiles}

\begin{document}

%%%%%%%%%%%%%%%%%%%%%%%%%%%%%%%%%%%%%%%%%%%%%%%%%%%%%%%%%%%%%%%%%%%%%%%%%%%%%%%%%%%%%%%%%%%%%%%%%%%%%%%%%
\section{Kompaktní prostory}
%%%%%%%%%%%%%%%%%%%%%%%%%%%%%%%%%%%%%%%%%%%%%%%%%%%%%%%%%%%%%%%%%%%%%%%%%%%%%%%%%%%%%%%%%%%%%%%%%%%%%%%%%
\subsection{Definice kompaktního prostoru}
\hspace{1.2mm}
\noindent
Metrický prostor $(X,d)$ je kompaktní, pokud každá posloupnost v něm obsahuje konvergentní podposloupnost.

%%%%%%%%%%%%%%%%%%%%%%%%%%%%%%%%%%%%%%%%%%%%%%%%%%%%%%%%%%%%%%%%%%%%%%%%%%%%%%%%%%%%%%%%%%%%%%%%%%%%%%%%%
\subsection{Tvrzení o podprostoru kompaktního prostoru}
\hspace{1.2mm}
\noindent
Podprostor kompaktního prostoru je kompaktní právě když je uzavřený.

\vspace{5mm}
\noindent
\textbf{Důkaz:} 
\begin{enumerate}
    \item Buď $Y$ uzavřený podprostor kompaktního $X$ a buď $(y_n)_n$ podposloupnost v $Y$. Jako posloupnost v X má podposloupnost s limitou
    a z uzavřenosti je tato limita v $Y$.
    \item Nechť $Y$ není uzavřená. Potom existuje posloupnost $(y_n)_n)$ v $Y$ konvergentní v $X$ taková, že $y = lim_n y_n \notin Y$. Potom $(y_n)_n$
    nemůže mít podposloupnost konvergentní v $Y$ protože každá její podposloupnost konverguje k $y$.
\end{enumerate}

%%%%%%%%%%%%%%%%%%%%%%%%%%%%%%%%%%%%%%%%%%%%%%%%%%%%%%%%%%%%%%%%%%%%%%%%%%%%%%%%%%%%%%%%%%%%%%%%%%%%%%%%%
\subsection{Tvrzení o uzavřenosti podprostoru}
\hspace{1.2mm}
\noindent
Buď $(X,d)$ libovolný metrický prostor a
buď podprostor $Y \subseteq X$ kompaktní. Potom $Y$ je uzavřený v $(X,d)$.

\vspace{5mm}
\noindent
\textbf{Důkaz:} Nechť $(y_n)_n$ posloupnost v $Y$ konverguje v $X$ k limitě $y$. Potom každá podposloupnost $(y_n)_n$ konverguje k 
$y$ a tedy je $y \in Y$.
\begin{center}
    Metrický prostor $(X,d)$ je omezený, jestliže pro nějaké $K$ platí, že 
    \[\forall x,y \in X : d(x,y) < K.\]
\end{center}

%%%%%%%%%%%%%%%%%%%%%%%%%%%%%%%%%%%%%%%%%%%%%%%%%%%%%%%%%%%%%%%%%%%%%%%%%%%%%%%%%%%%%%%%%%%%%%%%%%%%%%%%%
\subsection{Tvrzení o omezenosti kompaktního prostoru}
\hspace{1.2mm}
\noindent
Každý kompaktní prostor je omezený.

\vspace{5mm}
\noindent
\textbf{Důkaz:} Zvolme $x_1$ libovolně a $x_n$ tak, aby $d(x_1,x_n) > n$. Posloupnost $(x_n)_n$ nemá konvergentní podposloupnost; kdyby $x$
byla limita takové podposloupnosti, bylo by pro dost velké $n$ nekonečně mnoho členů této podposloupnosti blíže k $x_1$ než $d(x_1,x_n)+1$, což je spor.

%%%%%%%%%%%%%%%%%%%%%%%%%%%%%%%%%%%%%%%%%%%%%%%%%%%%%%%%%%%%%%%%%%%%%%%%%%%%%%%%%%%%%%%%%%%%%%%%%%%%%%%%%
\subsection{Věta o součinu kompaktních prostorů}
\hspace{1.2mm}
\noindent
Součin konečně mnoha kompaktních prostorů je kompaktní.

\vspace{5mm}
\noindent
\textbf{Důkaz:} Stačí dokázat pro součin dvou prostorů.

Buďte $(X,d_1), (X, d_2)$ kompaktní a buď $((x_n,y_n))_n$ posloupnost v $X \times Y$. 
Zvolme konvergentní podposloupnost $(x_{k_n})_n$ posloupnosti $(x_n)_n$ a konvergentní podposloupnost $(y_{k_{l_n}})_n$ posloupnosti $(y_{k_n})_n$.

Potom je 
\[((x_{k_{l_n}},y_{k_{l_n}}))_n\]
konvergentní podposloupnost posloupnosti $((x_n,y_n))_n$.

\begin{center}
    Kompaktní interval v $\mathbb{E}_n$: součin intervalů $\left<a_i,b_i\right>$
\end{center}

%%%%%%%%%%%%%%%%%%%%%%%%%%%%%%%%%%%%%%%%%%%%%%%%%%%%%%%%%%%%%%%%%%%%%%%%%%%%%%%%%%%%%%%%%%%%%%%%%%%%%%%%%
\newpage
\subsection{Věta : podprostor euklidovského prostoru je kompaktní právě když je omezený a uzavřený}
\hspace{1.2mm}
\noindent
Podprostor euklidovského prostoru $\mathbb{E}_n$ je kompaktní právě když je uzavřený a omezený.

\vspace{5mm}
\noindent
\textbf{Důkaz:} 
\begin{enumerate}
    \item Že je uzavřený a omezený už víme.
    \item Buď nyní $Y \subseteq \mathbb{E}_n$ omezený a uzavřený. Jelikož je omezený, je pro dostatečně velký kompaktní interval
    \[Y \subseteq J^n \subseteq \mathbb{E}_n.\]
    $J^n$ je kompaktní jako součin intervalů $\left<a_i,b_i\right>$, a jelikož je $Y$ uzavřený v $\mathbb{E}_n$ je též uzavřený
    v $J^n$ a tedy kompaktní.
\end{enumerate}

%%%%%%%%%%%%%%%%%%%%%%%%%%%%%%%%%%%%%%%%%%%%%%%%%%%%%%%%%%%%%%%%%%%%%%%%%%%%%%%%%%%%%%%%%%%%%%%%%%%%%%%%%
\subsection{Tvrzení: obraz spojitého zobrazení je kompaktní}
\hspace{1.2mm}
\noindent
Buď $f: (X,d) \to (Y, d')$ spojité zobrazení a buď $A \subseteq X$ kompaktní. Potom je $f[A]$ kompaktní.


\vspace{5mm}
\noindent
\textbf{Důkaz:} Buď $(y_n)_n$ posloupnost v $f[A]$. Zvolme $x_n \in A$ tak, aby $y_n = f(x_n)$. Buď $(x_{k_n})_n$ konvergentní podposloupnost
Potom je $(y_{k_n})_n = (f(x_{k_n}))_n$ konvergentní podposloupnost $(x_n)_n$.

%%%%%%%%%%%%%%%%%%%%%%%%%%%%%%%%%%%%%%%%%%%%%%%%%%%%%%%%%%%%%%%%%%%%%%%%%%%%%%%%%%%%%%%%%%%%%%%%%%%%%%%%%
\subsection{Tvrzení: každá spojitá funkce na kompaktním prostoru nabýva maxima i minima}
\hspace{1.2mm}
\noindent
Buď $(X,d)$ kompaktní. Potom každá spojitá funkce $f:(X,d)\to \mathbb{R}$ nabývá maxima i minima.

\vspace{5mm}
\noindent
\textbf{Důkaz:} Buď $Y = f[X] \subseteq \mathbb{R}$ kompaktní. Je to tedy omezená množina a musí mít supremum $M$ a infimum $m$. Zřejmě máme 
$d(m,Y) = d(M,Y) = 0$ a jelikož $Y$ je uzavřená, $m,M \in Y$. Víme, že spojitá $f$ je charakterizována tím, že všechny vzory uzavřených množin
jsou uzavřené. Nyní vidíme, že je-li definiční obor kompaktní, platí též, že obrazy uzavřených podmnožin jsou uzavřené. Z toho plyne následujíci:

%%%%%%%%%%%%%%%%%%%%%%%%%%%%%%%%%%%%%%%%%%%%%%%%%%%%%%%%%%%%%%%%%%%%%%%%%%%%%%%%%%%%%%%%%%%%%%%%%%%%%%%%%
\subsection{Věta o vzájemně jednoznačném spojitém zobrazení}
\hspace{1.2mm}
\noindent
Je-li $(X,d)$ kompaktní a je-li $f: (X,d) \to (Y,d')$ vzájemně jednoznačné spojité zobrazení, pak je
$f$ homeomorfismus.

\vspace{2mm}
\hspace{1.2mm}
{\small
Obecněji: Nechť $f:(X,d) \to (Y,d')$ je spojité zobrazení. Mějme potom $g: (X,d) \to (Z, d'')$ a
$h: (Y,d') \to (Z,d'')$ takové, že $h \circ f = g$. Potom je $h$ spojité.}

\vspace{5mm}
\noindent
\textbf{Důkaz:} Buď $B$ uzavřená v $Z$. Potom je $A = g^{-1}[B]$ uzavřená $\implies$ kompaktnost v $X$ $\implies$ $f[A]$ je kompaktní $\implies$ uzavřená v $Y$. Jelikož  je $f$ zobrazení na, máme $f[f^{-1}[C]] = C \forall C$. Proto je 
\[h^{-1}[B] = f[f^{-1}[h^{-1}[B]]] = f[(h \circ f)^{-1}[B]] = f[g^{-1}[B]] = f[A]\]
uzavřená.

%%%%%%%%%%%%%%%%%%%%%%%%%%%%%%%%%%%%%%%%%%%%%%%%%%%%%%%%%%%%%%%%%%%%%%%%%%%%%%%%%%%%%%%%%%%%%%%%%%%%%%%%%
\subsection{Definice cauchyovské posloupnosti $(x_n)_n$}
\hspace{1.2mm}
\noindent
Posloupnost $(x_n)_n$ v $(X,d)$ je \textbf{Cauchyovská}, jestliže
\[ \forall \epsilon > 0 \exists n_0: m,n \geq n_0 \implies d(x_m, x_n) < \epsilon \]

%%%%%%%%%%%%%%%%%%%%%%%%%%%%%%%%%%%%%%%%%%%%%%%%%%%%%%%%%%%%%%%%%%%%%%%%%%%%%%%%%%%%%%%%%%%%%%%%%%%%%%%%%
\subsection{Tvrzení o konvergenci cauchyovské posloupnosti}
\hspace{1.2mm}
\noindent
Nechť má Cauchyovská posloupnost konvergentní podposloupnost. Potom posloupnost konverguje k limitě
podposloupnosti.

\vspace{5mm}
\noindent
\textbf{Důkaz:} Nechť je $(x_n)_n$ Cauchyovská a nechť $lim_{n}x_{k_n} = x.$ Buď $d(x_m,x_n) < \varepsilon$ pro $m,n \geq n_1$ 
a $d(x_{k_n},x) \leq \varepsilon$ pro $n \geq n_2$. Položíme-li $n_0 = max(n_1,n_2)$, máme pro $n \geq n_0$ (protože $k_n \geq n$)
\[d(x_n,x) \leq d(x_n,x_{k_n}) + d(x_{k_n},x) < 2\varepsilon.\]

%%%%%%%%%%%%%%%%%%%%%%%%%%%%%%%%%%%%%%%%%%%%%%%%%%%%%%%%%%%%%%%%%%%%%%%%%%%%%%%%%%%%%%%%%%%%%%%%%%%%%%%%%
\subsection{Definice úplného metrického prostoru}
\hspace{1.2mm}
\noindent
Metrický prostor $(X,d)$ je \textbf{úplný}, pokud v něm každá Cauchyovská posloupnost konverguje.

%%%%%%%%%%%%%%%%%%%%%%%%%%%%%%%%%%%%%%%%%%%%%%%%%%%%%%%%%%%%%%%%%%%%%%%%%%%%%%%%%%%%%%%%%%%%%%%%%%%%%%%%%
\subsection{Tvrzení: Podprostor úplného prostoru je úplný právě když je uzavřený}
\hspace{1.2mm}
\noindent
Podprostor úplného je úplný, právě když je uzavřený.

\vspace{5mm}
\noindent
\textbf{Důkaz:} 
\begin{enumerate}
    \item Buď $Y \subseteq (X,d)$ uzavřený. Buď $(y_n)_n$ Cauchyovská v $Y$. Potom je Cauchyovská 
    a tedy konvergentní v $X$ a kvůli uzavřenosti je limita v $Y$.
    \item Nechť $Y$ není uzavřený. Potom existuje posloupnost $(y_n)_n$ v $Y$ konvergentní v $X$ taková, že $lim_n y_n \notin Y$.
    Potom je $(y_n)_n$ Cauchyovská v $X$ a jelikož je vzálenost stejná, též v $Y$. Ale v $Y$ nekonverguje.
\end{enumerate}

%%%%%%%%%%%%%%%%%%%%%%%%%%%%%%%%%%%%%%%%%%%%%%%%%%%%%%%%%%%%%%%%%%%%%%%%%%%%%%%%%%%%%%%%%%%%%%%%%%%%%%%%%
\subsection{Tvrzení: Každý kompaktní prostor je úplný}
\hspace{1.2mm}
\noindent
Každý kompaktní prostor je úplný.

\vspace{5mm}
\noindent
\textbf{Důkaz:} Cauchyovská posloupnost má podle kompaktnosti konvergentní podposloupnost a tedy konverguje.

%%%%%%%%%%%%%%%%%%%%%%%%%%%%%%%%%%%%%%%%%%%%%%%%%%%%%%%%%%%%%%%%%%%%%%%%%%%%%%%%%%%%%%%%%%%%%%%%%%%%%%%%%
\subsection{Lemma o cauchyovské posloupnosti}
\hspace{1.2mm}
\noindent
Posloupnost $(x_{1}^{1}, ... , x_{1}^{n}), (x_{1}^{2},...,x_{n}^{2}), ...,(x_{1}^{k},...,x_{n}^{k}),...$
je Cauchyovská v $\prod_{i=1}^{n}(X_i, d_i)$ právě když každá z posloupností $(x_{i}^{k})_k$ je
Cauchyovská v $(X_i, d_i)$.

\vspace{5mm}
\noindent
\textbf{Důkaz:} $\implies$ plyne bezprostředně z toho, že $d_i(u_i,v_i) \leq d((u_j)_j,(v_j)_j).$

$\Leftarrow$: Nechť je každá $(x_i^k)_k$ Cauchyovská. Pro $\varepsilon > 0$ a $i$ zvolme $k_i$ tak,
aby pro $k,l \geq k_i$ bylo $d_i(x_i^k, x_i^l) < \varepsilon.$ Potom pro $k,l \geq$ $\text{max}_i k_i$ máme 
\[d((x_1^k,...,x_n^k),(x_1^l,...,x_n^l)) < \varepsilon.\]

%%%%%%%%%%%%%%%%%%%%%%%%%%%%%%%%%%%%%%%%%%%%%%%%%%%%%%%%%%%%%%%%%%%%%%%%%%%%%%%%%%%%%%%%%%%%%%%%%%%%%%%%%
\subsection{Věta: Součin úplných prostorů je úplný }
\hspace{1.2mm}
\noindent
Součin úplných prostorů je úplný. Speciálně, $\mathbb{E}_n$ je úplný.

\subsubsection{Důsledek}
\hspace{1.2mm}
\noindent
Podprostor $Y$ euklidovského prostoru $\mathbb{E}_n$ je úplný, právě když je uzavřený.

\end{document}