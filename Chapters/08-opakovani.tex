\documentclass[../main.tex]{subfiles}

\begin{document}

%%%%%%%%%%%%%%%%%%%%%%%%%%%%%%%%%%%%%%%%%%%%%%%%%%%%%%%%%%%%%%%%%%%%%%%%%%%%%%%%%%%%%%%%%%%%%%%%%%%%%%%%%
\section{Opakování}
%%%%%%%%%%%%%%%%%%%%%%%%%%%%%%%%%%%%%%%%%%%%%%%%%%%%%%%%%%%%%%%%%%%%%%%%%%%%%%%%%%%%%%%%%%%%%%%%%%%%%%%%%
\subsection{Riemannův integrál v jedné proměnné}

\noindent
\textbf{Rozdělení} intervalu $\left<a,b\right>$ je posloupnost 
\[P : a = t_0 < t_1 < \cdot \cdot \cdot < t_{n-1} < t_n = b.\]

\noindent
\textbf{Zjemnění} rozkladu $P$ je rozklad $P'$ takový, že
	\[P' : a = t'_0 < t'_1 < \cdot \cdot \cdot < t'_{m-1} < t'_m = b\]
	\[\text{kde }\{t_j: j = 1,...,n-1\}\subseteq \{t'_j : j = 1,...,m-1\}.\] {
}

\noindent
\textbf{Jemnost} rozkladu $P$ je $$\mu(P) = \max_j(t_j-t_{j-1}).$$

\noindent
Pro omezenou $f:J=\left<a,b\right> \rightarrow \mathbf{R} $ a $P$ definujeme \textbf{dolní} a \textbf{horní součty}
\begin{align*}
    s(f,P) = & \sum^n_{j=1} m_j(t_j-t_{j-1}) \text{ resp.}\\
    S(f,P) = & \sum^n_{j=1} M_j(t_j-t_{j-1})
\end{align*}
kde

\[m_j = \inf\{f(x) : t_{j-1} \leq x \leq t_j\}, M_j = \text{sup}\{f(x) : t_{j-1} \leq x \leq t_j\}.\]

\begin{itemize}
    \item Pokud $P'$ zjemňuje $P$ dostáváme
    \[s(f,P) \leq s(f,P') \text{ a } S(f,P) \geq S(f,P')\]
    \item Pro každá dvě $P_1, P_2$ je 
    \[s(f,P_1) \leq S(f, P_2).\]
\end{itemize}

%%%%%%%%%%%%%%%%%%%%%%%%%%%%%%%%%%%%%%%%%%%%%%%%%%%%%%%%%%%%%%%%%%%%%%%%%%%%%%%%%%%%%%%%%%%%%%%%%%%%%%%%%
\subsubsection{Riemannův integrál}
\noindent
\textbf{Dolní/horní Riemannův integrál} $f$ přes $\left<a,b\right>$ jsou výrazy:
$$\underline{\int}^b_{ a} f(x)dx = \text{sup}\{s(f,P) : P \text{ rozdělení}\} \qquad \text{a}\qquad
\overline{\int}^b_{ a} f(x)dx = \text{inf}\{S(f,P) : P \text{ rozdělení}\}$$ 

Jsou-li si rovny, mluvíme o \textbf{Riemannově integrálu} funkce $f$ přes $\left<a,b\right>$:
\[\int^b_a f(x) dx\]

\noindent
%%%%%%%%%%%%%%%%%%%%%%%%%%%%%%%%%%%%%%%%%%%%%%%%%%%%%%%%%%%%%%%%%%%%%%%%%%%%%%%%%%%%%%%%%%%%%%%%%%%%%%%%%
\subsubsection{Tvrzení o existenci Riemannova integrálu}
\hspace{1.2mm}
Riemannův integrál $\int^b_a f(x) dx$ existuje právě když $\forall \varepsilon > 0 \exists$ rozdělení $P$ takové, že
\[S(f,P) - s(f,P) < \varepsilon.\]

\vspace{5mm}
\noindent
\textbf{Důkaz:} 
\begin{enumerate}
	\item[$\Rightarrow$:] Nechť $\int^b_a f(x) dx$ existuje a nechť $\varepsilon > 0$. Potom existují rozdělení $P_1$ a $P_2$ takové, že
    \begin{center}
        \begin{tabular}{ c c c }
            $S(f,P_1) < \int^b_a f(x) dx + \frac{\varepsilon}{2}$ & a & $s(f,P_2) > \int^b_a f(x) dx - \frac{\varepsilon}{2}$  \\
        \end{tabular}
    \end{center}
    Potom je pro společné zjemnění $P$ těch dvou $P_1,P_2$
    \[S(f,P) - s(f,P) < \int^b_a f(x)dx + \frac{\varepsilon}{2} - \int^b_a f(x)dx + \frac{\varepsilon}{2} = \varepsilon.\]
    \item[$\Leftarrow$:] Nechť druhé tvrzení platí. Zvolme $\varepsilon > 0 : S(f,P) - s(f,P) < \varepsilon.$ Potom je 
    \[\overline{\int}^b_a f(x)dx \leq S(f,P) < s(f,P) + \varepsilon \leq \underline{\int}^b_a f(x)dx + \varepsilon,\]
    a jelikož $\varepsilon$ bylo libovolně malé, vidíme, že $\overline{\int}^b_a f(x)dx = \underline{\int}^b_a f(x)dx.$
\end{enumerate}
\noindent


%%%%%%%%%%%%%%%%%%%%%%%%%%%%%%%%%%%%%%%%%%%%%%%%%%%%%%%%%%%%%%%%%%%%%%%%%%%%%%%%%%%%%%%%%%%%%%%%%%%%%%%%%
\subsubsection{Věta: Existence Riemannova integrálu pro spojité funkce v $\mathbb{R}$}
\hspace{1.2mm}
Pro každou spojitou $f : \left<a,b\right> \rightarrow \mathbb{R}$ Riemannův integrál $\int^b_a f$ existuje.
\vspace{5mm}

\noindent
\textbf{Důkaz:} Pro $\varepsilon > 0 $ zvolme $\delta > 0$ tak, aby 
\[\forall x,y : |x-y| < \delta \implies |f(x) - f(y)| < \frac{\varepsilon}{b-a}.\]
Je-li $\mu(P) < \delta$ máme $t_j-t_{j-1} < \delta$ pro všechna $j$, a tedy
\[M_j - m_j = \text{sup}\{f(x) : t_{j-1} \leq x \leq t_j\} - \text{inf}\{f(x) : t_{j-1} \leq x \leq t_j\} \leq\]
\[\leq \text{sup}\{|f(x) - f(y)| : t_{j-1} \leq x,y \leq t_j\} \leq \frac{\varepsilon}{b-a}\]
takže
\[S(f,P) - s(f,P) = \sum (M_j - m_j)(t_j-t_{j-1})\leq\]
\[\leq \frac{\varepsilon}{b-a}\sum (t_j-t{j-1}) = \frac{\varepsilon}{b-a}(b-a) = \varepsilon.\]

%%%%%%%%%%%%%%%%%%%%%%%%%%%%%%%%%%%%%%%%%%%%%%%%%%%%%%%%%%%%%%%%%%%%%%%%%%%%%%%%%%%%%%%%%%%%%%%%%%%%%%%%%
\subsubsection{Integrální věta o střední hodnotě}
\hspace{1.2mm}
Buď $f: \left< a,b \right> \to \mathbb{R}$ spojitá. Potom existuje $c \in \left< a,b \right>$
takové, že
\[ \int_{a}^{b} f(x) \,dx = f(c)(b-a)\]

\vspace{5mm}
\noindent
\textbf{Důkaz:} Položme $m = \min \{ f(x)|a \leq x \leq b \}$ a $M = \max \{ f(x)|a \leq x \leq b \} $
Zřejmě
\[ m(b-a) \leq \int_{a}^{b} f(x) \,dx \leq M(b-a) \]
Existuje tedy $K$ takové, že $m \leq K \leq M$ a $\int_{a}^{b} f(x) \,dx = K(b-a)$.
Jelikož $f$ je spojitá, existuje $c \in \left< a,b \right>$ takové, že $K = f(c)$.

%%%%%%%%%%%%%%%%%%%%%%%%%%%%%%%%%%%%%%%%%%%%%%%%%%%%%%%%%%%%%%%%%%%%%%%%%%%%%%%%%%%%%%%%%%%%%%%%%%%%%%%%%
\subsubsection{Základní věta analýzy}
\hspace{1.2mm}
Buď $f: \left< a,b \right> \to \mathbb{R}$ spojitá. Pro $x \in \left< a,b \right>$ definujeme
\[ F(x) = \int_{a}^{x} f(t) \,dt \]
Potom je $F'(x) = f(x)$

\vspace{5mm}
\noindent
\textbf{Důkaz:}
Pro $h\neq 0$ máme
\[ \frac{1}{h}(F(x+h) - f(x)) =\frac{1}{h}\left( \int_{a}^{x+h} f - \int_{a}^{x} f \right) =
\frac{1}{h} \int_{x}^{x+h} f = \frac{1}{h}f(x + \theta h)h = f(x + \theta h) \]

%%%%%%%%%%%%%%%%%%%%%%%%%%%%%%%%%%%%%%%%%%%%%%%%%%%%%%%%%%%%%%%%%%%%%%%%%%%%%%%%%%%%%%%%%%%%%%%%%%%%%%%%%
\subsubsection{Důsledky základní věty analýzy}
\hspace{1.2mm}
\begin{enumerate}
    \item Spojitá funkce $f : \left<a,b\right> \rightarrow \mathbb{R}$ má na intervalu $(a,b)$ primitivní funkci spojitou na $\left<a,b\right>$.
          Pro kteroukoli primitivní funkci $G$ funkce $f$ na $(a,b)$ spojitou na $\left<a,b\right>$ platí
          \[\int^b_a f(t)dt = G(b) - G(a).\]
    \item Integrální věta o střední hodnotě:
    \[F(b) - F(a) = \int^b_a f = f(c)(b-a) = F'(c)(b-a)\]
\end{enumerate}
\noindent

\end{document}
