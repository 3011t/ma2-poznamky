\documentclass[../main.tex]{subfiles}

\begin{document}

%%%%%%%%%%%%%%%%%%%%%%%%%%%%%%%%%%%%%%%%%%%%%%%%%%%%%%%%%%%%%%%%%%%%%%%%%%%%%%%%%%%%%%%%%%%%%%%%%%%%%%%%%
\section{Riemannův integrál ve více proměnných}
%%%%%%%%%%%%%%%%%%%%%%%%%%%%%%%%%%%%%%%%%%%%%%%%%%%%%%%%%%%%%%%%%%%%%%%%%%%%%%%%%%%%%%%%%%%%%%%%%%%%%%%%%
\subsection{Pomocné definice}
\hspace{1.2mm}
V $\mathbb{E}_n:$ \textbf{Kompaktní interval} (n-rozměrný) je 
\[J = \left<a_1,b_1\right> \times \cdot \cdot \cdot \times \left<a_n,b_n\right>\]
\begin{center}
(interval nebo cihla)
\end{center}
\hspace{1.2mm}
\textbf{Rozdělení intervalu} $J$ je posloupnost $P = (P^1,...P^n)$ rozdělení:
\[P^j : a_j = t_{j0} < t_{j1} < \cdot \cdot \cdot < t_{j,n_j-1} < t_{j,n_j} = b_j\]

\noindent
\hspace{1.2mm}
\textbf{Intervalům}
\[\left<t_{1,i_1},t_{1,i_1+1}\right> \times \cdot \cdot \cdot \times \left<t_{n,i_n},t_{n,i_n+1}\right>\]
říkáme cihly rozdělení $P$ a $\mathcal{B}(P)$ je množina všech cihel rozdělení $P$. Je to skoro disjunktní rozklad intervalu $J$.
Různe cihly z $\mathcal{B}(P)$ se totiž setkávají jen v podmnožinách okrajů, tedy v množinách objemu 0. Máme tedy:
\[\textbf{vol}(J) = \sum \{\textbf{vol}(B) : B \in \mathcal{B}(J)\}.\]

\textbf{Jemnost rozdělení}
\textit{Diametr} intervalu $J = \left<r_1,s_1\right> \times \cdot \cdot \cdot \times \left<r_n,s_n\right>$ je 
\[\textbf{diam}(J) = \max_i (s_i - r_i)\]
\textit{Jemnost rozdělení} $P$ je 
\[\mu(P) = \max \{\textbf{diam}(B) : B \in \mathcal{B}(P)\}.\]

\textbf{Zjemnění}

Rozdělení $Q = (Q^1,...Q^n) $ zjemňuje rozdělení $P = (P^1,...,P^n)$ jestliže každé $Q^j$ zjemňuje $P^j$.

Zjemňění $Q$ rozdělení $P$ vytváří rozdělení
\begin{tabular}{c c c}
     $Q_B$ & cihel & $B \in \mathcal{B}(P)$  
\end{tabular}
a máme skoro disjunktní sjednocení 
\[\mathcal{B}(Q) = \bigcup \{\mathcal{B}(Q_B) : B \in \mathcal{B}(P)\}.\]

Každá dvě rozdělení $P,Q$ $n-$rozměrného kompaktního intervalu $J$ mají spoločné zjemnění:
\vspace{5mm}
\noindent
\textbf{$\implies$} Je dána omezená $f: J \rightarrow \mathbb{R}$ na $n$-rozměrném kompaktním intervalu $J$ a $B \subseteq J$ je 
$n$-rozměrný kompaktní podinterval intervalu $J$. Položme
\[m(f,B) = \inf\{f(\textbf{x}) : \textbf{x} \in B\} \text{ a}\]
\[M(f,B) = \sup\{f(\textbf{x}) : \textbf{x} \in B\}.\]

\textbf{Fakt:} $m(f,B) \leq M(f,B)$ a je-li $C \subseteq B$, pak 
\[m(f,C) \geq m(f,B) \text{ a } M(f,C) \leq M(f,B).\]

Pro rozdělení $P$ intervalu $J$ a omezenou funkci $f : J \rightarrow \mathbb{R}$ definujeme 
\[s_J(f,P) = \sum \{m(f,B) \cdot \textbf{vol}(B) : B \in \mathcal{B}(P)\},\]
\[S_J(f,P) = \sum \{M(f,B) \cdot \textbf{vol}(B) : B \in \mathcal{B}(P)\}.\]

\textbf{Obecné pozorování:}

$f: X \rightarrow \mathbb{R}$ je omezená, $X = \bigcup X_i, X_i = \bigcup X_{ij}$ jsou konečná skoro disjunktní sjednocení.
\[M_i = \sup\{f(x) : x \in X_i\},\]
\[M_{ij} = \sup\{f(x) : x \in X_{ij}\}\]
\newpage
Triviálně $M_{ij} \leq M_i$ ( $M_i$ je horní mez množiny $\{f(x) : x \in X_{ij}\}$).

Tedy:
\[\sum M_i \textbf{vol}(X_i) = \sum_i M_i \sum_j \textbf{vol}(X_{ij}) = \]
\[= \sum_{ij}M_i \textbf{vol}(X_{ij}) \geq \sum_{ij} M_{ij} \textbf{vol}(X_{ij})\]

a podobně pro infima.

\textbf{Tvrzení:} Nechť $Q$ zjemňujě $P$. Potom
\begin{center}
    \begin{tabular}{c c c}
        $s(f,Q) \geq s(f,P)$ & a & $S(f,Q) \leq S(f,P)$
    \end{tabular}
\end{center}

\vspace{5mm}
\noindent
\textbf{Důkaz:} Použijeme předchozí pozorování pro $\{X_i | i\} = \mathcal{B}(P)$, $\{X_{ij} | j\} = \mathcal{B}(Q_B)$ a samozřejmě
i pro $\{X_{ij} | ij\} = \mathcal{B}(Q).$

\textbf{Tvrzení:} Pro libovolná dvě rozdělení $P,Q$ intervalu $J$ máme $s(f,P) \leq S(f,Q)$.

\vspace{5mm}
\noindent
\textbf{Důkaz:} Jelikož je triviálně $s(f,P) \leq S(f,P),$ použitím společného zjemnění $R$ rozdělení $P,Q$ dostaneme
\[s(f,P) \leq s(f,R) \leq S(f,R) \leq S(f,Q).\]

Množina $\{s(f,P) | P \text{ rozdělení}\}$ je tedy shora omezená a můžeme definovat dolní Riemannův integrál funkce $f$ přes $J$ jako
\[\underline{\int}_J f(\textbf{x})d\textbf{x} = \sup\{s(f,P) | P \text{ rozdělení}\};\]
podobně definujeme horní Riemannův integrál
\[\overline{\int}_J f(\textbf{x})d\textbf{x} = \inf\{S(f,P) | P \text{ rozdělení}\}.\]

Jsou-li si rovny, máme Riemannův integrál funkce $f$ přes $J$; značení:
\begin{center}
\begin{tabular}{c c c}
    $\int_J f(\textbf{x})d\textbf{x}$ nebo prostě & $\int_J f $\\
\end{tabular}
\end{center}

\textbf{Jiné značení:}
\[\int_J f(x_1,...,x_n)dx_1,...x_n\]
nebo
\[\int_J f(x_1,...,x_n)dx_1 dx_2\cdot \cdot \cdot dx_n\]

\textbf{Tvrzení:} Riemannův integrál $\int_J f(\textbf{x})d\textbf{x}$ existuje právě když $\forall \varepsilon > 0 \exists$ rozdělení $P : $
\[S_J(f,P) - s_J(f,P) < \varepsilon.\]
\vspace{5mm}
\noindent
\textbf{Důkaz:} nerovnost dává
\[S_J(f,P) < \varepsilon + s_J(f,P)\]
a z toho máme
\[\overline{\int} \leq S_J(f,P) < \varepsilon + s_J(f,P) \leq \varepsilon + \underline{\int} \leq \varepsilon + \overline{\int};\]
kde $\varepsilon$ může být libovolně malé.
%%%%%%%%%%%%%%%%%%%%%%%%%%%%%%%%%%%%%%%%%%%%%%%%%%%%%%%%%%%%%%%%%%%%%%%%%%%%%%%%%%%%%%%%%%%%%%%%%%%%%%%%%
\subsection{Tvrzení o existenci Riemannova integrálu}
\hspace{1.2mm}
Riemannův integrál $\int_{J} f(\mathbf{x}) \,d\mathbf{x}$ existuje právě když
$\forall \epsilon > 0$ existuje rozdělení $P$ takové, že
\[ S_J(f,P) - s_J(f,P) < \epsilon \]

\vspace{5mm}
\noindent
\textbf{Důkaz:}
Nerovnost dává \[ S_J(f,P) < \epsilon + s_J(f,P) \]
z toho dostaneme
\[ \overline{\int} \leq S_J(f,P) \leq \epsilon + s_J(f,P) \leq \epsilon + \underline{\int} \leq
\epsilon + \overline{\int}\]
pro libovolně malé $\epsilon$

%%%%%%%%%%%%%%%%%%%%%%%%%%%%%%%%%%%%%%%%%%%%%%%%%%%%%%%%%%%%%%%%%%%%%%%%%%%%%%%%%%%%%%%%%%%%%%%%%%%%%%%%%
\subsection{Věta: Každá spojitá funkce na n-rozměrnem kompaktním intervalu má Riemannův integrál}
\hspace{1.2mm}
Každá spojitá funkce $f: J \to \mathbb{R}$ na $n$-rozměrném kompaktním intervalu má Riemannův integrál
$\int_{J}f$.

\vspace{5mm}
\noindent
\textbf{Důkaz:}
V $\mathbb{E}_n$ budeme používat vzdálenost $\sigma$ definovanou předpisem
\[ \sigma (\mathbf{x}, \mathbf{y}) = \max_{i} |x_i - y_i| \]
Jelikož je $f$ stejnoměrně spojitá, můžeme pro $\epsilon > 0$ zvolit $\delta > 0$ takové, že
\[ \sigma (\mathbf{x}, \mathbf{y}) < \delta \Rightarrow
|f(\mathbf{x} - f(\mathbf{y}))| < \frac{\epsilon}{\textrm{vol}(J)} \]

\noindent
\hspace{1.2mm}
Připomeňme si jemnost $\mu (P)$. Je-li $\mu (P) < \delta$ je $\textrm{diam}(B) < \delta$ pro všechny
$ B \in \mathcal{B}(P) $ a tedy
\begin{align*}
    M(f, B) - m(f, B) &= \sup\{ f(\mathbf{x})|\mathbf{x} \in B \} -
    \inf\{ f(\mathbf{x})|\mathbf{x} \in B\}\leq\\
    &\leq \sup\{ |f(\mathbf{x}) - f(\mathbf{y})|: \mathbf{x}, \mathbf{y} \in B \}
    = \frac{\epsilon}{\textrm{vol}(J)}
\end{align*}
takže
\begin{align*}
    S(f,P) - s(f,P) &= \sum \{ (M(f,B) - m(f,B))\cdot \textrm{vol}(B)|B\in \mathcal{B}(P) \}\leq\\
    &\leq \frac{\epsilon}{\textrm{vol}(J)}\sum \{ \textrm{vol}(B)| B\in \mathcal{B}(P) \}
    = \frac{\epsilon}{\textrm{vol}(J)}\textrm{vol}(J) = \epsilon
\end{align*}

%%%%%%%%%%%%%%%%%%%%%%%%%%%%%%%%%%%%%%%%%%%%%%%%%%%%%%%%%%%%%%%%%%%%%%%%%%%%%%%%%%%%%%%%%%%%%%%%%%%%%%%%%
\subsection{Fubiniova věta}
\hspace{1.2mm}
Vezměme součin $J = J' \times J'' \subseteq \mathbb{E}_{m+n}$ intervalů $J' \subseteq \mathbb{E}_m$,
$J'' \subseteq \mathbb{E}_n$. Nechť existuje
\[ \int_{J} f(\mathbf{x}, \mathbf{y}) \,d\mathbf{xy} \]
a nechť pro každé $\mathbf{x} \in J'$, resp. $\mathbf{y} \in J''$, existuje
\[ \int_{J'} f(\mathbf{x}, \mathbf{y}) \,d\mathbf{x}
\hspace{5mm}\textrm{resp.}\hspace{5mm}
\int_{J''} f(\mathbf{x}, \mathbf{y}) \,d\mathbf{y} \]
Potom je
\[ \int_J f(\mathbf{x}, \mathbf{y}) \,d\mathbf{xy} =
\int_{J'} \left( \int_{J''} f(\mathbf{x}, \mathbf{y}) \,d\mathbf{y} \right) \,d\mathbf{x} = 
\int_{J''} \left( \int_{J'} f(\mathbf{x}, \mathbf{y}) \,d\mathbf{x} \right) \,d\mathbf{y}\]

\vspace{10mm}
\noindent
Tedy ve dvou proměnných
\[ \int_{J} f = \int_{a_1}^{b_1} \left( \int_{a_2}^{b_2} f(x,y) \,dy \right) \,dx \]
ve třech proměnných
\[ \int_{J} f =
\int_{a_1}^{b_1} \left(
\int_{a_2}^{b_2} \left(
\int_{a_3}^{b_3} f(x_1, x_2, x_3) \,dx_3 \right) \,dx_2 \right) \,dx_1 \]
a obecně
\[ \int_{J} f =
\int_{a_1}^{b_1} \left(
\int_{a_2}^{b_2} \left(
\dots \left(
\int_{a_n}^{b_n} f(x_1, x_2, ..., x_n) \,dx_n \right) \dots \right) \,dx_2 \right) \,dx_1 \]

\vspace{5mm}
\noindent
\textbf{Důkaz:}
Položme
\[ F(\mathbf{x}) = \int_{J''} f(\mathbf{x}, \mathbf{y}) \,d\mathbf{y} \]
Dokážeme, že $\int_{J'} F$ existuje a že
\[ \int_{J} f = \int_{J'} F \]
Zvolme rozdělení $P$ intervalu $J$ tak, aby
\[ \int f - \epsilon \leq s(f,P) \leq S(f,P) \leq \int f + \epsilon \]
Toto rozdělení je tvořeno rozděleními $P'$ intervalu $J'$ a $P''$ intervalu $J''$. Máme
\[ \mathcal{B}(P) = \{ B' \times B'' | B' \in \mathcal{B}(P'), B'' \in \mathcal{B}(P'') \} \]
a každá cihla $P$ se objeví jako právě jedno $B' \times B''$. Potom je
\[ F(\mathbf{x}) \leq \sum_{B''\in \mathcal{B}(P'')}
\max_{\mathbf{y} \in B''} f(\mathbf{x}, \mathbf{y}) \cdot \textrm{vol}B'' \]
a tedy
\begin{align*}
    S(F, P')
    &\leq \sum_{B' \in \mathcal{B}(P')} \max_{\mathbf{x}\in B'}
    \left( \sum_{B'' \in \mathcal{B}}(P'') \max_{\mathbf{y}\in B''}
    f(\mathbf{x}, \mathbf{y}) \cdot \textrm{vol}(B'')\right) \cdot \textrm{vol}(B') \leq\\
    &\leq \sum_{B' \in \mathcal{B}(P')} \sum_{B'' \in \mathcal{B}(P'')}
    \max_{(\mathbf{x}, \mathbf{y}) \in B' \times B''} f(\mathbf{x}, \mathbf{y})
    \cdot \textrm{vol}(B'') \cdot \textrm{vol}(B') \leq\\
    &\leq \sum_{B' \times B'' \in \mathcal{B}(P)} \max_{\mathbf{z}\in B' \times B''}
    f(\mathbf{z}) \cdot \textrm{vol}(B' \times B'') =\\
    = S(f,P)
\end{align*}
a podobně
\[ s(f,P) \leq s(F,P') \]
Máme tedy
\[ \int_{J} f - \epsilon \leq s(F,P') \leq \int_{J'} F \leq S(F,P) \leq \int_{J} f + \epsilon \]
a $\int_{J'} F$ je roven $\int_{J} f$.

%%%%%%%%%%%%%%%%%%%%%%%%%%%%%%%%%%%%%%%%%%%%%%%%%%%%%%%%%%%%%%%%%%%%%%%%%%%%%%%%%%%%%%%%%%%%%%%%%%%%%%%%%
\subsection{Lebesgueův integrál}
\hspace{1.2mm}
Riemannův integrál je intuitivně velmi uspokojivý a počítá to, co chceme, pokud tedy funguje.
Jeho užití má ale několik problémů:
\begin{itemize}
    \item Nemusí existovat i pro některé přirozeně definované funkce, nebo
    přinejmenším není snadno vidět, zda existuje.
    \item Nemůžeme provádět užitečné operace(limity, derivování) dost univerzálně.
\end{itemize}
\textbf{Lebesgueův integrál} je rozšíření Riemannova integrálu, kde můžeme dělat prakticky cokoliv,
za snadno zapamatelných podmínek.
\underline{Několik Lebesgueovských pravidel}:
\begin{enumerate}
    \item Je-li interval a Riemannův integrál $\int_{J} f$ existuje, shoduje se s Lebesgueovým.
    \item Pokud $ \int_{D_n} f $ f existuje pro $n = 1, 2, ...$, existuje i \[\int_{\bigcup D_n} f\]
    \item Pokud $ \int_{D} f_n $ existuje a posloupnost $(f_n)_n$ je monotónní, platí
    \[ \int_D \lim_n f_n = \lim_n \int_D f_n \]
    \item Pokud $\int_D f_n$ existuje a $|f_n|\leq g$ pro nějaké $g$ pro které existuje
    $\int_D g$, platí
    \[ \int_D \lim_n f_n = \lim_n \int_D f_n \]
    \item Je-li $D$ omezená, $|f_n(x)| \leq C$ a $\int_D f_n$ existují, platí
    \[ \int_D \lim_n f_n = \lim_n \int_D f_n \]
    \item Buď $U$ okolí bodu $t_0$ a $g$ takové, že existují $\int_D g$ a $\int_D f(t,x)\,dx$ a
    $\forall t \in U\textrm{\textbackslash}\{t_0\}: |f(t,x)|\leq g(x)($ potom
    \[ \int_D f(t_0, x)\,dx = \lim_{t\to t_0} \int_D f(t,x)\,dx \]
    \item Jestliže pro integrovatelnou $g$ platí
    \[ \left| \frac{\partial f(t,x)}{\partial t} \right| \leq g(x) \]
    a v nějakém okolí $U$ bodu $t_0$ všechno dává smysl(?), potom platí
    \[ \int_D \frac{\partial f(t_0,-)}{\partial t} = \frac{d}{dt} \int_D f(t_0, -)\]
\end{enumerate}


%%%%%%%%%%%%%%%%%%%%%%%%%%%%%%%%%%%%%%%%%%%%%%%%%%%%%%%%%%%%%%%%%%%%%%%%%%%%%%%%%%%%%%%%%%%%%%%%%%%%%%%%%
\subsection{Tietzeova věta}
\hspace{1.2mm}
Buď $Y$ uzavřený podprostor metrického prostoru $X$. Potom můžeme každou spojitou reálnou
funkci $f$ na $Y$ takovou, že $\forall x \in Y: a \leq f(x) \leq b$ rozšířit na stejně omezenou
spojitou funkci $g$ na $X$.

\end{document}