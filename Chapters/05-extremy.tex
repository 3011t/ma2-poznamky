\documentclass[../main.tex]{subfiles}

\begin{document}

%%%%%%%%%%%%%%%%%%%%%%%%%%%%%%%%%%%%%%%%%%%%%%%%%%%%%%%%%%%%%%%%%%%%%%%%%%%%%%%%%%%%%%%%%%%%%%%%%%%%%%%%%
\section{Extrémy}
\hspace{1.2mm}
\noindent

%%%%%%%%%%%%%%%%%%%%%%%%%%%%%%%%%%%%%%%%%%%%%%%%%%%%%%%%%%%%%%%%%%%%%%%%%%%%%%%%%%%%%%%%%%%%%%%%%%%%%%%%%
\subsection{Věta o hledání extrémů funkcí}
\hspace{1.2mm}
\noindent
Buďte $f,g_1, ... , g_k$ reálné funkce definované na otevřené množině $D \subseteq \mathbb{E}_n$.
Nechť mají spojité parciální derivace. Nechť je hodnost matice
\[ M = \begin{pmatrix}
\frac{\partial g_1}{\partial x_1} & \dots & \frac{\partial g_1}{\partial x_n}\\
\vdots & \ddots & \vdots\\
\frac{\partial g_k}{\partial x_1} & \dots & \frac{\partial g_k}{\partial x_n}\\
\end{pmatrix}\]
maximální, tedy $k \leq n$, v každém bodě oboru $D$.

\noindent
\hspace{1.2mm}
Jestliže funkce $f$ nabývá v bodě $\mathbf{a} = (a_1, ... , a_n)$ lokálního extrému podmíněného vazbami
\[ g_i(x_1, ... , x_n) = 0 \forall i \in \{ 1, ... , k \} \]
pak existují čísla $\lambda _1, ... , \lambda _k$ taková, že $\forall i \in {1, ... , n}$ platí
\[ \frac{\partial f(\mathbf{a})}{\partial x_i} +
\sum_{j=1}^{k}\lambda_j \cdot \frac{\partial g_j(\mathbf{a})}{\partial x_i} = 0 \]

\vspace{5mm}
\noindent
\textbf{Důkaz:} Matice $M$ má hodnost $k$ právě když aspoň jedna její $k\times k$ podmatice $M$ je regulární (a tedy má nenulový determinant). Dejme tomu,
\[ 0 \neq \begin{vmatrix}
\frac{\partial g_1}{\partial x_1} & \dots & \frac{\partial g_1}{\partial x_k}\\
\vdots & \ddots & \vdots\\
\frac{\partial g_k}{\partial x_1} & \dots & \frac{\partial g_k}{\partial x_k}\\
\end{vmatrix}\]
Potom podle věty o implicitních funkcích máme okolí bodu $\textbf{a}$ funkce $\phi_i(x_{k+1},...,x_n)$
se spojitými parciálními derivacemi takové, že (pišme $\Tilde{\textbf{x}}$ pro $(x_{k+1},...,x_n))$
\[g_i(\phi_1(\Tilde{\textbf{x}}),...,\phi_k(\Tilde{\textbf{x}}),\Tilde{\textbf{x}}) = 0 \text{ pro } i=1,...,k.\]
tedy lokální maximum nebo minimum funkce $f(\textbf{x}$) v $\textbf{a}$ podmíněné danými vazbami dává lokální maximum či minimum (nepodmíněné) funkce
\[F(\Tilde{\textbf{x}}) = f(\phi_1(\Tilde{\textbf{x}}),...,\phi_k(\Tilde{\textbf{x}}),\Tilde{\textbf{x}}),\]
v $\Tilde{\textbf{a}}$, a tedy je 
\[\frac{\partial F(\Tilde{\textbf{a}})}{\partial x_i} = 0 \text{ pro } i = k+1,...,n,\]
to jest, podle řetízkového pravidla
\[\sum^k_{r=1}\frac{\partial f(\textbf{a})}{\partial x_r}\cdot \frac{\partial \phi_r(\Tilde{\textbf{a}})}{\partial x_i} + \frac{\partial f(\textbf{a})}{\partial x_i} \text{ pro } i = k+1,...,n.\]
Derivováním konstantní $g_i(\phi_1(\Tilde{\textbf{x}},...,\phi_k(\Tilde{\textbf{x}}),\Tilde{\textbf{x}}) = 0$ dostaneme pro $j = 1,...,k$
\[\sum^k_{r=1}\frac{\partial g_j(\textbf{a})}{\partial x_r}\cdot \frac{\partial \phi_r(\Tilde{\textbf{a}})}{\partial x_i} + \frac{\partial g_j(\textbf{a})}{\partial x_i} \text{ pro } i = k+1,...,n.\]
Dále použijeme znovu vlastnost toho, že determinant je nenulový. Vzhledem k hodnosti matice má systém lineárních rovnic
\[\frac{\partial f(\textbf{a})}{\partial x_i} + \sum^n_{j=1}\lambda_j\cdot\frac{\partial g_j(\textbf{a})}{\partial x_i} = 0, i = 1,...,k\]
jediné řešení $\lambda_1,...,\lambda_k.$ To jsou rovnosti z tvrzení, ale jen pro $i \leq k$. Musíme ještě dokázat, že to platí i pro $i > k$. 

\begin{align*}
\frac{\partial f(\textbf{a})}{\partial x_i} + & \sum^n_{j=1}\lambda_j\cdot\frac{\partial g_j(\textbf{a})}{\partial x_i}=\\
= & - \sum^k_{r=1} \frac{\partial f(\textbf{a})}{\partial x_r}\cdot\frac{\partial \phi_r(\Tilde{\textbf{a}})}{\partial x_i} -
\sum^k_{j=1}\lambda_j\cdot\sum^k_{r=1}\frac{\partial g_j(\textbf{a})}{\partial x_r}\cdot\frac{\partial \phi_r(\Tilde{\textbf{a}})}{\partial x_i} =\\
= & - \sum^n_{r=1}\left(\frac{\partial f(\textbf{a})}{\partial x_i}+\sum^n_{j=1}\lambda_j\cdot\frac{\partial g_j(\textbf{a})}{\partial x_i}\right)\frac{\partial \phi_r(\Tilde{\textbf{a}})}{\partial x_i}=\\
= & - \sum^n_{r=1}0\cdot\frac{\partial \phi_r(\Tilde{\textbf{a}})}{\partial x_i}=0.
\end{align*}
%%%%%%%%%%%%%%%%%%%%%%%%%%%%%%%%%%%%%%%%%%%%%%%%%%%%%%%%%%%%%%%%%%%%%%%%%%%%%%%%%%%%%%%%%%%%%%%%%%%%%%%%%
\subsection{Definice Regulárního zobrazení}
\hspace{1.2mm}
\noindent
Buď $U \subseteq \mathbb{E}_n$ otevřená a nechť mají $f_i$ pro $i \in {1, ... , n}$
spojité parciální derivace. Výsledné zobrazení
\[ \mathbf{f} = (f_1, ... , f_n): U \to \mathbb{E}_n \]
je \textbf{regulární}, jestliže
\[ \forall \mathbf{x} \in U: \frac{D(\mathbf{f})}{D(\mathbf{x})}(\mathbf{x}) \neq 0 \]


%%%%%%%%%%%%%%%%%%%%%%%%%%%%%%%%%%%%%%%%%%%%%%%%%%%%%%%%%%%%%%%%%%%%%%%%%%%%%%%%%%%%%%%%%%%%%%%%%%%%%%%%%
\subsection{Tvrzení o obrazu regulární funkce}
\hspace{1.2mm}
\noindent
Je-li $\mathbf{f}: U \to \mathbb{E}_n$ regulární, je obraz $\mathbf{f}[V]$ každé otevřené podmnožiny
$V \subseteq U$ otevřený.

\vspace{5mm}
\noindent
\textbf{Důkaz:} Vezměme $f(\textbf{x}^0) = \textbf{y}^0.$ Definujeme $\textbf{F} : V \times \mathbb{E}_n \rightarrow \mathbb{E}_n$ předpisem
\[F_i(\textbf{x},\textbf{y}) = f_i(\textbf{x}) - y_i.\]
Potom je $\textbf{F}(\textbf{x}^0,\textbf{y}^0) = \textbf{0}$
a $\frac{D(\textbf{F})}{D(\textbf{x})} \neq 0$, 
a tedy můžeme použít větu o IF a dostaneme 
$\delta > 0$ a $\Delta > 0 : \forall \textbf{y} : ||\textbf{y} - \textbf{y}^0|| < \delta \exists \textbf{x} : ||\textbf{x} - \textbf{x}^0|| < \Delta$ a 
$F_i(\textbf{x},\textbf{y}) = f_i(\textbf{x}) - y_i = 0$. To znamená, že máme $\textbf{f}(\textbf{x}) = \textbf{y}$ (pozor, $y_i$ jsou zde proměnné, $x_j$ hledané funkce a
\[\Omega(\textbf{y}^0,\delta) = \{\textbf{y} : ||\textbf{y} - \textbf{y}^0 || < \delta \} \subseteq \textbf{f}[V].\]

%%%%%%%%%%%%%%%%%%%%%%%%%%%%%%%%%%%%%%%%%%%%%%%%%%%%%%%%%%%%%%%%%%%%%%%%%%%%%%%%%%%%%%%%%%%%%%%%%%%%%%%%%
\subsection{Tvrzení o inverzi regulárního zobrazení}
\hspace{1.2mm}
\noindent
Buď $\mathbf{f}: U \to \mathbf{E}_n$ regulární zobrazení. Potom $\forall \mathbf{x}^0 \in U \exists$
otevřené okolí $V$ takové, že restrikce $\mathbf{f}|V$ je bijekce. Navíc, zobrazení
$\mathbf{g}: f[V] \to \mathbb{E}_n$ inverzní k $\mathbf{f}|V$ je regulární.

\vspace{5mm}
\noindent
\textbf{Důkaz:} Znovu použijeme zobrazení $\textbf{F} = (F_1,...,F_n)$, kde $F_i(\textbf{x},\textbf{y}) = f_i(\textbf{x})-y_i$. Pro dost malé 
$\Delta > 0$ máme právě jedno $\textbf{x} = \textbf{g}(\textbf{y})$ takové, že $\textbf{F}(\mathbf{g}(\textbf{y}),\textbf{y}) = 0$ a $||\textbf{x} - \textbf{x}^0|| < \Delta$.
Toto $\textbf{g}$ má navíc spojité parciální derivace. Máme
\[D(id) = D(\textbf{f}\circ\textbf{g}) = D(\textbf{f})\cdot D(\textbf{g}).\]
Podle řetízkového pravidla (a věty o násobení determinantů) je 
\[\frac{D(\textbf{f})}{D(\textbf{x})}\cdot\frac{D(\textbf{g})}{D(\textbf{y})} = \det D(\textbf{f})\cdot \det D(\textbf{g}) = 1\]

a tedy je pro každé $\textbf{y} \in \textbf{f}[V], \frac{D(\textbf{g})}{D(\textbf{y})}(\textbf{y}) \neq 0$.

%%%%%%%%%%%%%%%%%%%%%%%%%%%%%%%%%%%%%%%%%%%%%%%%%%%%%%%%%%%%%%%%%%%%%%%%%%%%%%%%%%%%%%%%%%%%%%%%%%%%%%%%%
\subsubsection{Důsledek tvrzení o inverzi regulárního zobrazení}
\hspace{1.2mm}
\noindent
Prosté regulární zobrazení $\mathbf{f}: U \to \mathbb{E}_n$ má regulární inverzi
$\mathbf{g}: \mathbf{f}[U] \to \mathbb{E}_n$

\end{document}