\documentclass[11pt]{article}

\usepackage[czech]{babel}
\usepackage{a4wide}
\usepackage[utf8]{inputenc}
\usepackage[T1]{fontenc}
\usepackage{fancyhdr}
\usepackage{amssymb}
\usepackage{amsthm}
\usepackage{amsmath}
\usepackage{mathtools}
\usepackage{mleftright}
\usepackage{subfig}

% forcing footnotes to be at the very bottom
\usepackage[bottom]{footmisc}

\usepackage{hyperref}
\usepackage{titlesec}
%This has to be the last
\usepackage{subfiles}

\usepackage{geometry}
\geometry{
    a4paper,
    total={170mm,257mm},
    right=20mm,
    left=20mm,
    top=30mm,
    bottom=20mm,
}

\DeclareMathOperator{\rank}{rank}
\DeclareMathOperator{\Span}{span}

\newtheoremstyle{nontheoremstyle}{1em}{1em}{}{}{\bfseries}{:}{.5em}{}
\newtheoremstyle{theoremstyle}{1em}{1em}{\it}{}{\bfseries}{:}{.5em}{}

\theoremstyle{nontheoremstyle}
\newtheorem*{definition}{Definice}
\newtheorem*{example}{Příklad}
\renewenvironment{proof}{{\noindent\bfseries Důkaz:}}{\qed}
\newtheorem*{intuition}{Intuice}
\newtheorem*{remark}{Poznámka}
\newtheorem*{consequence}{Důsledek}
\newtheorem*{observation}{Pozorování}

% ošklivý hack k tomu, aby environment 'definitionnodot' neměl na konci . nebo :
% hodí se, když chceme Dělat něco jako „Definice (Riemannův integrál) je funkce...“
\newtheoremstyle{nontheoremstylenodot}{1em}{1em}{}{}{\bfseries}{}{.3em}{}
\theoremstyle{nontheoremstylenodot}
\newtheorem*{definitionnodot}{Definice}

\theoremstyle{theoremstyle}
\newtheorem*{theorem}{Věta}
\newtheorem*{lemma}{Tvrzení}

\titleformat{\section} {\normalfont\fontsize{16}{15}\bfseries}{\thesection}{1em}{}
\titleformat{\subsection} {\normalfont\fontsize{14}{15}\bfseries}{\thesubsection}{1em}{}
\titleformat{\subsubsection} {\normalfont\fontsize{12}{15}\bfseries}{\thesubsubsection}{1em}{}

\pagestyle{fancy}
\fancyhf{}
%\lhead{Viktor Soukup, Lukáš Salak}
\rhead{Matematická analýza II}
\fancyfoot{}
\fancyfoot[R]{\thepage}

\begin{document}

\begin{titlepage}
    \begin{center}
        \vspace*{1cm}
            
        \Huge
        \textbf{Matematická analýza II}
            
        \vspace{0.5cm}
        \LARGE
        Stručné výpisky
        \\

        z materiálů prof. Pultra

        \vspace{5mm}
        
        Zimní semestr 2020/2021
        
        \vspace{1.5cm}
            
        \textbf{Viktor Soukup, Lukáš Salak, Tomáš Sláma}
        
        \vfill
        \flushright
        \normalsize
        Revize : Mgr. Karel Král,\\
        Verze 2.1\\
        \today
        
    \end{center}
\end{titlepage}

\tableofcontents
\clearpage


\subfile{Chapters/01-metricke-prostory.tex}
\subfile{Chapters/02-parcialni-derivace.tex}
\subfile{Chapters/03-kompaktni-prostory.tex}
\subfile{Chapters/04-implicitni-funkce.tex}
\subfile{Chapters/05-extremy.tex}
\subfile{Chapters/06-objemy-obsahy.tex}
\subfile{Chapters/07-stejnomerna-spojitost.tex}
\subfile{Chapters/08-opakovani.tex}
\subfile{Chapters/09-multivar-riemann.tex}

\vfill
\begin{center}
\LARGE
\textbf{The End}
\end{center}

\end{document}
